\pdfbookmark[0]{Abstract}{Abstract}
\addchap*{Abstract}

Twitter's ability to connect users around a given topic provides insights into the complex mechanisms that grant positions of influence to a subset of users. This paper focuses on clustering a collection of Twitter topic networks using an interpretable approach centred on the platform's asymmetric relationships. This method consists of two general processes: it starts by identifying the structural profiles of the network users from a representation of the network based on the presence of directed subgraphs of 2-4 nodes and then we create \textit{embeddings} of the network using the previous profiles created and clusters are established within the collection. The applicability of the proposed method is shown by analysing 75 real networks generated around \textit{Trending Topics} in Mexico and discussing the identified user profiles from the standpoint of the social power dynamics they reflect.