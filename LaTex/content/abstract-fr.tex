\pdfbookmark[0]{Résumé}{Résumé}
\addchap*{Résumé}

La capacité de Twitter à relier les utilisateurs autour d'un sujet donné permet de comprendre les mécanismes complexes qui confèrent des positions d'influence à un sous-groupe d'utilisateurs. Cet article se concentre sur le regroupement d'une collection de réseaux thématiques Twitter à l'aide d'une approche interprétable centrée sur les relations asymétriques de la plateforme. Cette méthode est composée par deux processus généraux : on commence par identifier les profils structurels des utilisateurs du réseau à partir d'une représentation du réseau basée sur la présence de sous-graphes dirigés de 2-4 nodes et ensuite on crée des \textit{embeddings} de ce réseau en utilisant les profils créés précédemment et ensuite on établit des \textit{clusters} à l'intérieur de la collection. L'applicabilité de la méthode proposée est illustrée par l'analyse de 75 réseaux réels générés autour de \textit{Trending Topics} au Mexique et par la discussion des profils d'utilisateurs identifiés du point de vue de la dynamique du pouvoir social qu'ils reflètent.