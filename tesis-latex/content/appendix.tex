
\chapter{Apéndice}
\label{chapter:appendix}

\subsubsection{Homofilia}

En sociología se denomina homofilia (del griego «amor a los iguales») a la tendencia de las personas por la atracción a sus homónimos. Esta atracción puede ser respecto a distintos atributos como edad, género, creencias, educación, estrato social, etc.

\subsubsection{Función Biyectiva}

Una función es biyectiva es aquella que es a la vez inyectiva y suprayectiva. Es decir, una función entre los elementos de dos conjuntos, donde cada elemento de un conjunto se empareja con exactamente un elemento del otro conjunto, y cada elemento del otro conjunto se empareja con exactamente un elemento del primer conjunto.

Formalmente, dada una función $f$

$ {\begin{array}{rccl}f:&X&\longrightarrow &Y\\&x&\longmapsto &y=f(x)\end{array}} $

Es biyectiva si para todo $y\in Y$ existe un único $x \in X$ al que la función evaluada en $x$ es igual a $y$.

\subsubsection{Línea base}
\label{sec:appendix:baseline}

Utilizando el árbol de decisión de \textit{Himelboim et. al} \cite{himelboim_classifying_2017} se clasificó el conjunto de datos de redes temáticas de Twitter para establecer una línea base. En este caso el agrupamiento que resulta no es es capaz de capturar la complejidad de la colección, y la mayoría de las redes quedan en un solo grupo ($clustered$).

\begin{table}[h]
\begin{center}
    \csvautotabular{csv/baseline1.csv}
    \caption{Resultado del agrupamiento realizado utilizando el árbol de decisión de \cite{himelboim_classifying_2017} para el conjunto de redes temáticas.}
\end{center}
\end{table}

\begin{table}[h]
\begin{center}
    \csvautotabular{csv/baseline2.csv}
    \caption{Resultado del agrupamiento realizado utilizando el árbol de decisión de \cite{himelboim_classifying_2017} para el conjunto de redes temáticas.}
\end{center}
\end{table}