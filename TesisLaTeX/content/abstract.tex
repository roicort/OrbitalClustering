% **************************************************
% Abstract en inglés
% **************************************************
\pdfbookmark[0]{Abstract}{Abstract}
\addchap*{Abstract}
\label{sec:abstract}

La aplicación de aprendizaje no supervisado para el agrupamiento (Clustering) de nodos en una red es un problema que han sido estudiado ampliamente, la agrupación de redes completas por otro lado, es un problema que aún ha de ser investigado debido a la complejidad de encontrar una medida de distancia o hacer comparaciones entre redes. Recientemente el concepto de roles estructurales ha sido propuesto junto a sus características topológicas para estudiar la composición de una red. En este trabajo de tesis se propone una metodología para obtener un embedding basado en el la firma orbital de los usuarios en una red, con el fin de realizar el agrupamiento de redes temáticas de Twitter utilizando aprendizaje no supervisado a partir de dicha representación vectorial basada en características estructurales.

{\vspace{5mm}\textbf{\textit{Keywords ---}} Graphlets, Órbitas, Embeddings, Clustering, Redes Sociales, Roles Estructurales} 