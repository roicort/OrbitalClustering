% **************************************************
% Abstract en inglés
% **************************************************
\pdfbookmark[0]{Abstract}{Abstract}
\addchap*{Abstract}
\label{sec:abstract}

La aplicación de aprendizaje no supervisado para el agrupamiento (Clustering) de nodos en una red es un problema que han sido estudiado ampliamente, la agrupación de redes completas por otro lado, es un problema que no se ha explorado ampliamente debido a la complejidad de encontrar una medida de distancia o hacer comparaciones entre redes. Recientemente el concepto de roles estructurales ha sido propuesto junto a sus características topológicas para estudiar la composición de una red. En este trabajo de tesis se propone una metodología para realizar un agrupamiento de redes temáticas de Twitter utilizando aprendizaje no supervisado sobre un representación (embedding) a partir de la caracterización de los roles estructurales de los usuarios dentro de la red.

{\vspace{5mm}\textbf{\textit{Keywords ---}} Redes Sociales, Clustering, Grafos, Roles Estructurales $\ldots$ Keyword5} 