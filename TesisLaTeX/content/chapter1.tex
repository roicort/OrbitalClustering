\chapter{Introducción}
\label{sec:intro}

%Qué, por qué y para qué?

%Capitulo de Divulgación

En un mundo cada vez más conectado, las interacciones de los usuarios en los espacios digitales crean una inmensidad de conexiones que a la vez reflejan complejas estructuras sociales. Estudiar estas redes y sus estructuras es de gran interés para distintas disciplinas ya que permiten extraer información valiosa que permite comprender distintos procesos sociales, como pueden ser el flujo de información, las interacciones y jerarquías entre los usuarios. 

Dado que las redes sociales como Twitter generan intensos debates relacionados con cuestiones sociopolíticas clave y tienen una gran capacidad para proyectar diversos discursos en el ámbito público, es de particular interés para muchos científicos la configuración de dichas redes en Twitter. Esta plataforma de \textit{ microblogging} se ha señalado como una pieza crítica en la construcción de debates políticos y movimientos sociales \cite{barbera_understanding_2015} e incluso de gran influencia en la configuración de la opinión pública sobre temas de de salud \cite{sharevski_misinformation_2022}.

Una búsqueda en la plataforma especializada \textit{Science Direct} arroja más de 32,449 artículos que involucran estudios de redes en Twitter, con ángulos que van desde los mecanismos de creación de redes, hasta los ecosistemas de poder creados en torno al flujo de información. Dado que las redes sociales como Twitter generan intensos debates relacionados con cuestiones sociopolíticas clave y tienen una gran capacidad para proyectar diversos discursos en el ámbito público. A continuación se describen algunos ejemplos de estudios realizados sobre Twitter en distintas disciplinas. 

\begin{itemize}

    \item \textbf{Política y movimientos sociales.} Twitter ha demostrado ser un importante actor dentro de recientes movimientos sociales y políticos. Algunos ejemplos interesantes de estudios que se han hecho en Twitter son \textit{Misinformation warnings: Twitter’s soft moderation effects on COVID-19 vaccine belief echoes} donde Sharevski \textit{et al.} estudian los efectos de la moderación de Twitter en las creencias sobre la efectividad de las vacunas durante la pandemia de COVID-19 \cite{sharevski_misinformation_2022} y en \textit{Twitter: A useful tool for studying elections?} Ivon Gaber estudia la correlación entre la actividad en Twitter y el desempeño electoral de los candidatos del Partido Laborista y el Partido de la Independencia en el Reino Unido. \cite{gaber_twitter_2017}

    \item \textbf{Salud pública}. En cuestiones de salud pública, Twitter es un herramienta útil para modelar las concepciones que los usuarios tienen sobre ciertos temas. En especifico, Han \textit{et al.} propone una metodología para modelar las ideas y el marketing detrás del uso de cigarrillos electrónicos en Estados Unidos \cite{spiro_exploratory_2016}.

    \item \textbf{Economía}. En \textit{Twitter mood predicts the stock market}, Bollen \textit{et al.} utilizan economia del comportamiento (\textit{ behavioral economics}) y Twitter para predecir el estado de ánimo colectivo en Twitter y estudiar la correlación con el mercado de valores \cite{bollen_twitter_2011}. De manera similar, Aharon \textit{et al.} miden el impacto de las \textit{Twitter Uncertainty Measures (TMU \& TEU)} sobre criptomonedas \cite{aharon_twitter-based_2022}.

    \item \textbf{Psicología, marketing e \textit{influencers}}. En \textit{Hashtag homophily in twitter network: Examining a controversial cause-related marketing campaign}, publicado en \textit{Computers in Human Behavior}, Sifan Xua y Alvin Zhoub estudiaron redes de campañas de marketing controversiales para analizar la tendencia a la homofilia de los usuarios que utilizaron ciertos \#hashtags. Los resultados del estudio muestran que a pesar de que la discusión se dio principalmente dentro de los discursos de la campaña, los usuarios reaccionaron más fuertemente ante los \texit{influencers}. Además, la red de menciones de estos usuarios mostró una tendencia a la homofilia basada en los hashtags ideológicos y no conceptuales \cite{xu_hashtag_2020}.

    \item \textbf{Lingüística, noticias y \textit{fake news}}. En 2020, Medford \textit{et al.} analizaron los sentimientos colectivos en Twitter sobre la pandemia de COVID-19. La mitad de los Tweets expresaron miedo mientras que un tercio expresó sorpresa. Al analizar los Tweets más retuiteados, el contenido se enfocaba en las formas de transmisión, los esfuerzos de prevención y la cuarentena, mientras que el miedo disminuía. En la cohorte completa, el impacto económico y político de COVID-19 fue el tema más discutido \cite{medford_infodemic_2020}. 

    Los procesos por los que las \textit{fake news} se diseminan y afectan la conversación pública también pueden ser estudiados en Twitter. En \textit{Modeling the spread of fake news on Twitter} se propone que las noticias falsas se diseminan como una noticia ordinaria hasta que los usuarios se dan cuenta de la falsedad y eso se convierte en otra noticia \cite{murayama_modeling_2021}.

\end{itemize}

En 2017, Himelboim \textit{ et al.} \cite{himelboim_classifying_2017} se enfocaron en el estudio de redes temáticas en Twitter. Es decir, analizaron las interacciones que surgen entre usuarios de la plataforma cuando se aborda un tema específico. Su trabajo no utiliza aprendizaje automático, pero propone una serie de reglas que les permite caracterizar diferentes redes temáticas. Este problema es interesante porque busca distinguir, en un conjunto de redes, las distintas configuraciones estructurales que pueden surgir. De manera intuitiva, los autores tratan de establecer similitudes y diferencias entre redes, de forma que puedan compararlas y crear grupos. 

Debe señalarse que el problema de agrupamiento de redes implica distintos retos computacionales. Debido a la naturaleza de los grafos, no se puede utilizar directamente los métodos convencionales de aprendizaje automático, como \textit{K-Means} \cite{bejar_k-means_nodate}, sobre los mismos; es necesario primero crear una representación vectorial. Además, tratándose de un trabajo de exploración, la representación debería poder interpretarse para que los resultados tengan significado para especialistas en otras áreas. 

En este trabajo de tesis se propone una metodología que, organizada en dos etapas principales, permite estudiar redes temáticas en Twitter a partir de sus estructuras locales utilizando como base la idea de \textit{órbitas} \cite{sarajlic_graphlet-based_2016} en {\emph graphlets}. Dichas órbitas corresponden a los roles de nodos en la colección de todos los posibles grafos de cierto orden dado (típicamente se consideran sólo 2-5 nodos), conocidos como \textit{graphlets} y originados en estudios de bioinformática \cite{przulj_biological_2007}. Con estas órbitas, que se describen con detalle más adelante, este trabajo construye una representación vectorial (\textit{embedding}) con el objetivo final de realizar un agrupamiento que tome en cuenta los roles estructurales de usuarios.

Es importante mencionar que dicha metodología ha sido aceptada en distintos congresos y será publicada en la \textit{Mexican Conference on Pattern Recognition (Proceedings)}.

En el resto de este capítulo se presenta una descripción de términos importantes relacionados con Twitter. Después, se motiva el estudio de redes temáticas con una perspectiva de roles estructurales. Finalmente, se establecen los objetivos y la metodología de esta investigación. 

\section{Twitter} 

Cada medio digital en el que usuarios interactúan define los canales y las estructuras del flujo de información. Tanto las estructuras de flujo como las jerarquías sociales en una plataforma reflejan patrones interesantes que nos permiten entender la relación que existen entre las mismas. Uno de los ejemplos más claros dentro de los medios digitales y las redes sociales donde se dan este tipo de interacciones y jerarquías es Twitter. Twitter es un servicio de \textit{microblogging} y red social en la que los usuarios publican e interactúan con posts conocidos como “tweets" \cite{twitter_twittercom_nodate}. 

Un tweet es la unidad mínima de Twitter, se trata de un mensaje de hasta 280 caracteres, son públicamente visibles por defecto y cualquier usuario puede responder a los demás, creando de esta manera una discusión pública que se puede modelar con una red dirigida.

La forma en que se propaga la información en Twitter se asemeja a cómo se propaga la información en la vida real. Las comunicaciones humanas suelen caracterizarse por una asimetría entre los productores de información (medios de comunicación, empresas, personas influyentes, entre otros) y los consumidores de contenidos \cite{gabielkov_studying_2014}. El papel de los usuarios en la propagación de la información a través de la red está intrínsecamente relacionado con la topología de la misma. Entender estos roles puede proporcionar una valiosa visión de los debates públicos en la plataforma. 

A continuación se describen algunos términos relevantes para analizar el funcionamiento de Twitter. 

%Corregir comillas.

\paragraph{Trending Topics.}
Twitter hace un seguimiento de las frases, palabras y hashtags que se mencionan con mayor frecuencia y los publica bajo el título de "Trending Topic". Un hashtag es una etiqueta por convención entre los usuarios de Twitter para crear y seguir un hilo de discusión prefijando una palabra con el símbolo “\#”. Los Trending Topics ayudan a Twitter y a sus usuarios a entender lo que está ocurriendo en la red social e invitarles a unirse a la discusión \cite{twitter_twittercom_nodate}. Los Trending Topics se representan filtrados por país dependiendo de la configuración de la cuenta y son calculados en tiempo real a lo largo del día. 

\paragraph{Interacciones.} Las mayor parte de las interacciones dentro de Twitter corresponden a la práctica común de responder o reaccionar a un tweet \cite{kwak_what_2010}. Las más comunes están definidas por las siguientes acciones: 
\begin{itemize}
    \item RT que la abreviatura “retweet" es la práctica de replicar el tweet de otro usuario. El mecanismo de retweet permite a los usuarios difundir la información que deseen más allá del alcance de los seguidores del tweet original.

    \item ‘@‘ seguido de un identificador (username) se refiere a una mención y se utiliza para etiquetar y responder directamente a un usuario.
\end{itemize}

\paragraph{Red temática.} Una red temática es aquella que captura las interacciones anteriormente mencionadas dentro de un tema en especifico definido por un Trending Topic. Es decir, los nodos de la red representan usuarios que han escrito un tweet sobre un tema en tendencia (TT) y las aristas representan las interacciones entre ellos, ya sea un RT o una mención. Es importante mencionar que las aristas son dirigidas y representan el sentido de la interacción.

\section{Agrupamiento de redes temáticas}

Como se mencionó anteriormente, las redes temáticas pueden ser interesantes ya que contiene la configuración estructural de la discusión pública sobre un tema en especifico. Con esta motivación, Himelboim \textit{ et al.} propusieron un estudio de redes temáticas usando criterios que ellos mismos definieron con base en su experiencia desde el campo de la sociología. 

En su trabajo, estos autores hacen clasificación, aunque no en el sentido de aprendizaje automático, pues no utilizan datos etiquetados ni siguen una metodología basada en los datos. Más bien proponen que hay 6 clases importantes para el estudio de redes, que son: dividida, unificada, fragmentada, clusterizada, \textit{in hub-and-spoke} y \textit{out hub-and-spoke}. Después, utilizando distintas medidas de las redes crean un árbol de decisión para clasificar cada una en los grupos predefinidos, como se puede observar en \ref{fig:himelboim}.

 \begin{figure}[htbp]
   \centering
   \includesvg[width=0.95\textwidth]{figures/himelboim.svg}
    \caption{Árbol de decisión para clasificar redes temáticas en Twitter, propuesto por Himelboim \textit{ et al.} \cite{himelboim_classifying_2017}} % ampliar esta explicación!!!
    \label{fig:himelboim}
\end{figure}

Aunque este trabajo se considera una aportación importante al estudio de redes en Twitter, utilizar grupos predefinidos podría llevar consigo algunos problemas, como limitar la clasificación a sólo las categorías concebidas por los autores, desestimando otros criterios que permitirían diferenciar entre redes. Preguntas interesantes que pueden plantearse a partir de este trabajo son: ¿Es posible llevar a cabo un agrupamiento basado directamente en los datos? ¿De qué forma puede hacerse si además se requiere que los resultados sean interpretables? Quizá los algoritmos de aprendizaje automático no-supervisado para agrupamiento no son directamente una opción, pero extrayendo características de las redes para crear un \textit{embedding} podría ser una alternativa viable.

\section{Roles estructurales y \textit{graphlets}} 

Los roles estructurales han sido estudiados por distintas disciplinas desde hace algunos años. Un rol estructural en redes puede entenderse como las funciones que tiene un nodo dentro de un grafo. La importancia de estos roles estructurales reside en su correlación con las estructuras y jerarquías sociales así como su comportamiento. 

Desde distintas disciplinas se ha intentado mapear las estructuras en grafos a estructuras sociales. En \textit{Understanding Network Formation in Strategy Research} \cite{rose_kim_understanding_2016} se estudia la composición de la redes dentro del contexto de investigación sobre gestión estratégica y cómo estas impactan directamente dentro de las organizaciones (Ver Fig.\ref{fig:rosekim}).

 \begin{figure}[htbp]
   \centering
   \includesvg[width=1\textwidth]{figures/rosekimunderstanding.svg}
    \caption{Roles estructurales y su función según Kim \textit{et al.} \cite{rose_kim_understanding_2016}} % ampliar esta explicación!!!
    \label{fig:rosekim}
\end{figure}

Otro ejemplo muy interesante es el de \textit{Structural Holes and Good Ideas} \cite{burt_structural_2004}, donde se describe el mecanismo por el que la intermediación influye directamente en el capital social. Esto debido en gran parte a que la opinión y el comportamiento son más homogéneos dentro de los grupos que entre todos ellos, por lo que las personas que conectan grupos (puentes) están más familiarizadas con formas alternativas de pensar y comportarse. 

En la Fig. \ref{fig:broker} podemos observar un ejemplo en el que encontramos un puente entre dos grupos.  Estos nodos ($A$ y $B$) también pueden ser encontrados en la literatura con el nombre de \textit{brokers} y tienen una alta intermediación. La centralidad de intermediación es una medida de centralidad en grafos basada en los caminos más cortos. Formalmente se define como $$g(v)=\sum _{{s\neq v\neq t}}{\frac  {\sigma _{{st}}(v)}{\sigma _{{st}}}}$$ dónde $\sigma_{st}$ es el número total de caminos más cortos desde el nodo $s$ al nodo $t$ y
$\sigma_{st}(v)$ es el número de esos caminos que pasan por $v$ ( donde $v$ no es un nodo final)

 \begin{figure}[htbp]
   \centering
   \includesvg[width=0.5\textwidth]{figures/bridge.svg}
    \caption{En un grafo se conoce como puente a los nodos que conectan dos grupos, estos nodos tiene una alta intermediación ya que necesariamente por ellos pasan los caminos más cortos entre nodos de ambos grupos.}
    \label{fig:broker}
\end{figure}

Dada la relevancia que, en sociología, ha tenido el análisis de roles estructurales, en este trabajo exploramos la posibilidad de agrupar redes temáticas en Twitter basándonos en la idea de dichos roles. Para ello, utilizamos las órbitas de graphlets, que son diccionarios de grafos de orden fijo, descritos con mayor detalle en el capítulo \ref{chapter:graphlets}.

\section{Presentación del problema y objetivos}
\label{sec:intro:motivación}

Los objetivos de este trabajo son los siguientes: 

\paragraph{Objetivo general.}
Dada una colección de redes de Twitter definidas por la interacción de los usuarios sobre temas concretos (redes temáticas), agrupar redes dentro de la colección según el perfil de los usuarios que conforman cada red, tomando como base el rol estructural de los usuarios.

\paragraph{Objetivos específicos}


\begin{itemize}
    \item[OE1] Crear una colección de redes temáticas en Twitter en México.
    \item[OE2] Identificar perfiles de usuarios en las redes mediante una representación vectorial a nivel nodo, basada en la firma orbital de graphlets. 
    \item[OE3] Construir una representación vectorial para las redes temáticas basada en la caracterización de usuarios y roles estructurales, y usarla para agrupar las redes en la colección.
\end{itemize}

\subsection{Metodología}
\label{sec:intro:organización}
\begin{enumerate}
    \item[OE1] Crear una colección de redes temáticas en Twitter en México.
    \begin{enumerate}
        \item Determinar un criterio para elegir temas que permitan la construcción de redes.
        \item Descargar tweets con los criterios previamente determinados de tal manera que las redes temáticas puedan ser construidas.
        \item Preprocesar los datos y construir las redes a partir de la discusión pública.
        \item Guardar las redes en un formato apropiado para trabajar con la colección.
    \end{enumerate}
    \item[OE2] Identificar perfiles de usuarios en las redes mediante una representación vectorial a nivel nodo, basada en la firma orbital de graphlets. 
    \begin{enumerate}
        \item Calcular los graphlets y la firma orbital de cada nodo para cada red
         \item Llevar a cabo clustering usando la firma orbital de los nodos que se calculó en el paso anterior.
        \item Identificar los distintos perfiles de usuario que se distinguen de acuerdo a la firma orbital.
    \end{enumerate}
    \item[OE3] Construir una representación vectorial para las redes temáticas basada en la caracterización de usuarios y roles estructurales, y usarla para agrupar las redes en la colección.
    \begin{enumerate}
        \item Representar cada red de acuerdo al tipo de usuarios que emergen en la conversación. 
        \item Agrupar las redes temáticas utilizando la representación anterior, de modo que puedan identificarse grupos basados en un criterio interpretable: el rol estructural de los usuarios. 
    \end{enumerate}
\end{enumerate}


\section{Estructura del trabajo}
En el capítulo 2 revisaremos algunos conceptos útiles relacionados con el agrupamiento en grafos. Después en el capítulo 3, se discutirán los graphlets y las órbitas que pueden definirse a partir de ellos. Tomando como base los capítulos 2 y 3, el capítulo 4 describe la la metodología propuesta. Posteriormente, en el capítulo 4 se exponen los resultados de los experimentos realizados. Finalmente, en el capítulo 5 encontramos las conclusiones del trabajo, algunas consideraciones del mismo y el trabajo futuro propuesto.