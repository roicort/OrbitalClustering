% **************************************************
% Abstract en inglés
% **************************************************
\pdfbookmark[0]{Abstract}{Abstract}
\addchap*{Abstract}
\label{sec:abstract}

La capacidad de Twitter para conectar a los usuarios en torno a un tema determinado permite conocer los complejos mecanismos que otorgan posiciones de influencia a un subconjunto de usuarios. Este trabajo se centra en el agrupamiento de una colección de redes temáticas de Twitter mediante un enfoque interpretable centrado en las relaciones asimétricas de la plataforma. Nuestro método consiste en dos pasos generales: primero, identificamos los perfiles estructurales de los usuarios de la red a partir de una representación de la red basada en la presencia de subgrafos dirigidos de 2 a 4 nodos. A continuación, creamos \textit{embeddings} de la red utilizando los perfiles anteriores creados y establecemos grupos dentro de la colección. Mostramos la aplicabilidad del método propuesto analizando 75 redes reales generadas en torno a \textit{Trendings Topics} en México y discutiendo los perfiles de usuarios identificados desde el punto de vista de las dinámicas de poder social que reflejan.

{\vspace{5mm}\textbf{\textit{Keywords ---}} Graphlets, Órbitas, Embeddings, Clustering, Redes Sociales, Roles Estructurales} 