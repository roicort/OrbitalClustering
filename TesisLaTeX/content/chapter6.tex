
\chapter{Conclusiones}
\label{sec:conclusion}

Con el uso de modelos basados en redes en diferentes disciplinas del conocimiento, el agrupamiento en conjuntos de redes se vuelve una tarea muy importante. Sin embargo, no todos los métodos existentes proporcionan resultados que puedan traducirse fácilmente a nuevas interpretaciones de los datos. 

En este trabajo se presenta una alternativa para agrupar redes sociales. El método propuesto tiene dos etapas principales: detectar el perfil de usuarios con base en su firma orbital en graphlets, y agrupar las redes de acuerdo a la caracterización de usuarios que las conforman. Nuestro enfoque es interpretable y capaz de captar la estructura de la red mediante el uso de graphlets 

La metodología presentada utiliza algoritmos computacionales ampliamente conocidos con implementaciones eficientes que permiten el desarrollo de cada paso propuesto. De este modo, nuestro enfoque aprovecha la utilidad de los graphlets y de sus órbitas asociadas para capturar información sobre la estructura de una red y llevar a cabo tareas de agrupamiento.
 
Mostramos la utilidad de la metodología propuesta a través de una aplicación real con redes temáticas de Twitter. Encontramos que los perfiles establecidos en el primer paso del método nos dan información útil sobre las estructuras de la red y las dinámicas sociales dentro de ellas. Esta descripción de perfiles puede considerarse una extensión de trabajo propuesto en sociología que sólo consideraba triadas de nodos. El método también reconoce que un usuario puede tener varios roles dentro de la discusión sobre un cierto tema en Twitter. 

Consideramos que nuestro enfoque tiene al menos dos ventajas. En primer lugar, proporciona un método para agrupar redes temáticas de Twitter de forma explicable, capturando las diferencias entre ellas que van más allá de las métricas generales de la red. En segundo lugar, produce una caracterización de los usuarios de la red que puede ayudar a comprender la estructura, las relaciones y los patrones latentes creados por la compleja dinámica de Twitter. 

%En particular, ¿qué dice sobre la dinámica de las redes que se puedan agrupar reconociendo los perfiles? Es decir, ¿en qué sentido fue una aportación hacer estos perfiles y no directamente usar los graphlets? Creo que hay elementos al final de la discusión en la sección anterior que pueden sugerir un párrafo para escribir aquí como un apunte general hacia la aportación de esta propuesta.

Desde el punto de vista sociológico, la utilización de perfiles de usuario sobre las redes temáticas, permite explorar las interacciones y las dinámicas que surgen durante una conversación pública en Twitter. Como vimos en el primer capítulo, el análisis de este tipo de redes permite modelar y comprender fenómenos asociados a este tipo de discusiones. Seguramente aún quedan distintas posibilidades de análisis por explorar a partir de estudio de las órbitas.

Entre las líneas de trabajo futuro que se proponen, existe la posibilidad de explorar la generalidad de los perfiles de usuario detectados. Es decir, queda por hacer un análisis más detallado de estos perfiles para extender la discusión con herramientas y metodologías de otras áreas afines.

