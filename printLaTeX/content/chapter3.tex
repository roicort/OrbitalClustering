
\chapter{Graphlets, Órbitas y Roles Estructurales}
\label{chapter:graphlets}

\section{Graphlets}

%colección (diccionario) para contar cosas inducidas en una red

Los \textit{graphlets} pueden ser entendidos como una colección o diccionario de todos los posibles subgrafos de hasta un tamaño fijo $n$. Estos subgrafos pueden contarse de manera inducida en un red más grande. En teoría de grafos un subgrafo inducido es el subgrafo de otro grafo que se conforma a partir de un subconjunto de vértices y de todas las aristas incidentes a pares de vértices del subconjunto \cite{przulj_biological_2007}.

%De manera formal un \textit{graphlet} Sea G=(V,E) 

 \begin{figure}[htbp]
   \centering
   \includesvg[width=1\textwidth]{figures/smallgraphlets.svg}
    \caption{Graphlets de 2, 3 y 4 nodos.}
    \label{fig:small-graphlets}
\end{figure}

\section{Órbitas y firma orbital}

 \begin{figure}[htbp]
   \centering
   \includesvg[width=1\textwidth]{figures/orcaorbits.svg}
    \caption{Graphlets y órbitas no dirigidas de 2 a 5 nodos.}
    \label{fig:orcaorbits}
\end{figure}

Los graphlets pueden ser extendidos a la noción de órbitas que son las posiciones posibles de un nodo puede tomar en un graphlet dirigido. De manera formal, son grupos de simetría de nodos (automorfismos) \cite{sarajlic_graphlet-based_2016} que describen los distintos roles topológicos en los que un nodo puede participar dentro del Graphlet. (Ver Fig. \ref{fig:orbits})

Una firma orbital es el conteo de las posiciones orbitales de un nodo, el vector resultante de los conteos es la firma orbital del nodo. Este vector describe la topología del nodo y su vecindario y captura sus interconexiones hasta una distancia $n$ \cite{sarajlic_graphlet-based_2016}.

Originalmente las firmas orbitales se utilizaron en el contexto de biología para analizar redes e identificar grupos de nodos topológicamente similares que por lo tanto compartieran propiedades biológicas similares \cite{milenkovic_uncovering_2008} de tal manera en que se pudieran predecir propiedades biológicas de nodos no caracterizados.

\subsection{Relevancia de las Órbitas y su relación con los roles estructurales}

Los Graphlets fueron introducidos por primera vez dentro del contexto biológico con la idea de comparar grafos. Los autores crearon un diccionario de todos los posibles subgrafos con 2-5 nodos; mientras que una comparación basada en la distribución de grado cuenta el número de nodos unidos a las aristas i, los autores comparan los gráficos contando el número de nodos unidos al graphlet i. Posteriormente los graphlets se extendieron a grafos dirigidos a través de la noción de órbitas \citep{sarajlic_graphlet-based_2016}. No obstante la relevancia de los graphlets y las orbitas van más allá de la mera utilidad para comparar grafos, recientemente se ha sugerido que la presencia, o ausencia, de ciertas estructuras locales dentro de una red podría tener un impacto crítico en la estructura general de la red.

\subsection{Conteo de órbitas}

Relacionado con el conteo de graphlets en una red, Sarajlíc et al. (2016) \cite{sarajlic_graphlet-based_2016} propuso hacer el conteo de orbitas para un nodo. El conteo se representa en forma de un vector en ${R}^{n}$ donde el componente $n$ representa la cantidad de veces que el nodo $u$ aparece en la orbita $n$. Las órbitas permiten distinguir los diferentes roles que puede tener un nodo dentro de un mismo graphlet, por lo tanto el cálculo de la firma orbital de un nodo proporciona información sobre su grado generalizado, basado en el graphlet, y las diferentes formas en las que interactúa con sus vecinos. 

\subsection{Ejemplo Karate Club}

La red Karate Club estudiada por Wayne W. Zachary en  \cite{zachary_information_1977} describe las interacciones de 34 miembros de un club de karate de 1970 a 1972, periodo durante el cual surgió un conflicto entre el administrador John A. y el instructor Mr. Hi. y el club se dividió en dos grupos al rededor de cada uno de ellos. Esta red (Ver Fig.\label{fig:karateclub}) se convirtió un estándar para el estudio de algoritmos y se utiliza como referencia para el estudio de los mismos.

 \begin{figure}[htbp]
  \centering
  \includesvg[width=1.\textwidth]{figures/karate.svg}
    \caption{Red de Karate Club \cite{zachary_information_1977}. Los nodos más influyentes, Mr. Hi, John A. y sus respectivos vecinos a distancia 1 han sido coloreados.}
    \label{fig:karateclub}
\end{figure}

En la Fig. \ref{fig:karateorbits} podemos observar los embeddings para 4 nodos de la red Karate Club. Para contrastar los embeddings de distintos tipos de nodos en la red tomamos como referencia a los más influyentes conocidos como Mr. Hi y John A. y a los menos conectados, los nodos 9 y 17. En el caso de los nodos más influyentes podemos observar que comparten órbitas dominantes, que son las órbitas 16, 21 y 23 descritas en la Fig. \ref{fig:orcaorbits}. Por otro lado las orbitas dominantes del nodo 9 son las 17, 20 y 22 y del nodo 17 son las 4, 15 y 18. Es importante notar que las órbitas 16, 21 y 23 son órbitas centrales pertenecientes a graphlets de 5 nodos.

 \begin{figure}[htbp]
   \centering
   \includesvg[width=1\textwidth]{figures/karateorbits.svg}
    \caption{Comparación del conteo de órbitas normalizado para 4 usuarios de la red \textit{Karate Club}.}
    \label{fig:karateorbits}
\end{figure}