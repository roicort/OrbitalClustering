
\chapter{Apéndice}
\label{sec:appendix}

\section{Capítulo 1}
\label{sec:appendix:c2}

\subsubsection{Homofilia}

En sociología se denomina homofilia (del griego «amor a los iguales») a la tendencia de las personas por la atracción a sus homónimos. Esta atracción puede ser respecto a distintos atributos como edad, género, creencias, educación, estrato social, etc.

\subsubsection{Centralidad de Intermediación}

La distancia de intermediación o \textit{Betweenes Centrality} es una medida de centralidad basada en geodésicas o caminos más cortos \cite{wikipedia_betweenness_nodate}.

Formalmente esta definida como:

$$g(v)=\sum _{{s\neq v\neq t}}{\frac  {\sigma _{{st}}(v)}{\sigma _{{st}}}}$$

donde $\sigma_{st}$ es el número total de caminos mas cortos desde el nodo $s$ al nodo $t$ y $\sigma_{st}(v)$ es el número total de esos caminos que pasan a través de $v$ (dónde $v$ no es el nodo final de un camino).

\section{Capítulo 2}
\label{sec:appendix:c2}

\subsubsection{Función Biyectiva}

Una función es biyectiva es aquella que es a la vez inyectiva y suprayectiva. Es decir, una función entre los elementos de dos conjuntos, donde cada elemento de un conjunto se empareja con exactamente un elemento del otro conjunto, y cada elemento del otro conjunto se empareja con exactamente un elemento del primer conjunto.

Formalmente, dada una función $f$

$ {\begin{array}{rccl}f:&X&\longrightarrow &Y\\&x&\longmapsto &y=f(x)\end{array}} $

Es biyectiva si para todo $y$ de $Y$ existe un único $x$ de $X$ al que la función evaluada en $x$ es igual a $y$

$ \forall y\in Y\;:\quad \exists !\ x\in X\;/\quad f(x)=y $

\section{Capítulo 5}
\label{sec:appendix:c3}

\subsubsection{Línea base}
\label{sec:appendix:baseline}

Utilizando el árbol de decisión de \textit{Himelboim et. al} \cite{himelboim_classifying_2017} se clasificó el conjunto de datos de redes temáticas de Twitter para establecer una línea base. En este caso el agrupamiento no es óptimo ya que la mayoría de las redes quedan en un solo grupo.

\begin{table}[h]
\begin{center}
    \csvautotabular{csv/baseline1.csv}
    \caption{Resultado del agrupamiento realizado utilizando el árbol de decisión de \cite{himelboim_classifying_2017} para el conjunto de redes temáticas.}
\end{center}
\end{table}

\begin{table}[h]
\begin{center}
    \csvautotabular{csv/baseline2.csv}
    \caption{Resultado del agrupamiento realizado utilizando el árbol de decisión de \cite{himelboim_classifying_2017} para el conjunto de redes temáticas.}
\end{center}
\end{table}