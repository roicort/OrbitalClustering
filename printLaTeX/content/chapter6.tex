
\chapter{Conclusiones}
\label{sec:conclusion}

Con el uso de modelos basados en redes en diferentes disciplinas del conocimiento, el agrupamiento en conjuntos de redes se vuelve una tarea muy importante. Sin embargo, no todos los métodos existentes proporcionan resultados que puedan traducirse fácilmente a nuevas interpretaciones de los datos. 

En este trabajo se presenta una alternativa para agrupar redes sociales. El método propuesto tiene dos etapas principales: detectar el perfil de usuarios con base en su firma orbital en graphlets, y agrupar las redes de acuerdo a la caracterización de usuarios que las conforman.  

La metodología presentada utiliza algoritmos computacionales ampliamente conocidos con implementaciones eficientes que permiten el desarrollo de cada paso propuesto. De este modo, nuestro enfoque aprovecha la utilidad de los graphlets y de sus órbitas asociadas para capturar información sobre la estructura de una red y llevar a cabo tareas de agrupamiento.

%Nuestro enfoque es interpretable y capaz de captar la estructura de la red mediante el uso de graphlets.
%con un método que provee información sobre la similitud entre redes basado en perfiles de usuario y sus roles estructurales a través de representaciones vectoriales (embeddings)
 
Mostramos la utilidad de la metodología propuesta a través de una aplicación real con redes temáticas de Twitter. Encontramos que los perfiles establecidos en el primer paso del método nos dan información útil sobre las estructuras de la red y las dinámicas sociales dentro de ellas. Esta descripción de perfiles puede considerarse una extensión de trabajo propuesto en sociología que sólo consideraba triadas de nodos. El método también reconoce que un usuario puede tener varios roles dentro de la discusión sobre un cierto tema en Twitter. 

% Pasar a conclusiones 
Consideramos que nuestro enfoque tiene al menos dos ventajas. En primer lugar, proporciona un método para agrupar redes temáticas de Twitter de forma explicable, capturando las diferencias entre ellas que van más allá de las métricas generales de la red. En segundo lugar, produce una caracterización de los usuarios de la red que puede ayudar a comprender la estructura, las relaciones y los patrones latentes creados por la compleja dinámica de Twitter. 

Entre las líneas de trabajo futuro que se proponen, está la posibilidad de explorar la generalidad de los perfiles de usuario detectados. Además, queda por hacer un análisis más detallado de estos perfiles con un punto de vista interdisciplinario. Es decir, la discusión aún puede ser extendida con herramientas y metodologías de otras áreas afines, por ejemplo de las ciencias sociales.